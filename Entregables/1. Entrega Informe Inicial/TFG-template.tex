\documentclass[10pt,a4paper,twocolumn,twoside]{article}
\usepackage[utf8]{inputenc}
\usepackage[english]{babel}
\usepackage{multicol}
\usepackage{graphicx}
\usepackage{fancyhdr}
\usepackage{times}
\usepackage{titlesec}
\usepackage{multirow}
\usepackage{lettrine}
\usepackage{hyperref}
\usepackage[top=2cm, bottom=1.5cm, left=2cm, right=2cm]{geometry}
\usepackage[figurename=Fig.,tablename=TAULA]{caption}
\captionsetup[table]{textfont=sc}

\author{\LARGE\sffamily Segovia Barreales, Richard}
\title{\Huge{\sffamily Improving the usability of a video-see-through head-mounted display}}
\date{}

\newcommand\blfootnote[1]{%
  \begingroup
  \renewcommand\thefootnote{}\footnote{#1}%
  \addtocounter{footnote}{-1}%
  \endgroup
}

%
%\large\bfseries\sffamily
\titleformat{\section}
{\large\sffamily\scshape\bfseries}
{\textbf{\thesection}}{1em}{}

\begin{document}

\fancyhead[LO]{\scriptsize AUTOR: SEGOVIA BARREALES, RICHARD}
\fancyhead[RO]{\thepage}
\fancyhead[LE]{\thepage}
\fancyhead[RE]{\scriptsize EE/UAB TFG INFORMÀTICA: IMPROVING THE USABILITY OF A VIDEO-SEE-THROUGH HEAD-MOUNTED DISPLAY}

\fancyfoot[CO,CE]{}

\fancypagestyle{primerapagina}
{
   \fancyhf{}
   \fancyhead[L]{\scriptsize TFG EN ENGINYERIA INFORMÀTICA, ESCOLA D'ENGINYERIA (EE), UNIVERSITAT AUTÒNOMA DE BARCELONA (UAB)}
   \fancyfoot[C]{\scriptsize March 2018, Escola d'Enginyeria (UAB)}
}

\renewcommand{\headrulewidth}{0pt}
\renewcommand{\footrulewidth}{0pt}
\pagestyle{fancy}

\maketitle

\thispagestyle{primerapagina}


\blfootnote{$\bullet$ E-mail: richard.segovia@e-campus.uab.cat}
\blfootnote{$\bullet$ Menció en Computació}
\blfootnote{$\bullet$ Treball tutoritzat per: Felipe Lumbreras (Computació)}
\blfootnote{$\bullet$ Curs 2017/18}

\section{Introduction}

\lettrine[lines=3]{T}{he} past summer during my internship I developed a basic video-see-through viewer for a headset prototype developed in the CVC\footnote{Computer Vision Center located in the UAB campus}, the main goal is to improve this viewer making it expandable for future modules and reduce the adverse physical reactions that it produces can produce to the users\cite{disconfortReview}. It has to be said that this research project is focused mainly in the software, the prototypes and other hardware aspects will be out of our concerns and will be developed by others researchers of the CVC.

\section{OBJECTIVES}

A objectives tree \ref{arbolReferemcias} has been done to  
The current objectives have been extracted from this objective tree \ref{imagenArbol}.

First we will be developing a software that will be capable of capture an stream from a stereo camera and display it in real time. The software will show one camera stream in each half of the screen, also the headset will enable only to each eye to see one half of the screen, all of this will simulate the stereo effect, creating a depth effect. Another goal is to prepare this software to be easily expandable with new modules in the future.

Secondly, as is reported in \cite{disconfortReview}, using head mounted displays can cause a variety of adverse physical reactions, since this headset will be used while working, these symptoms will reduce the concentration and the effective working hours. For that reason reducing these adverse physical reactions is a priority. To archive this goal, we will try to apply two techniques: 

\begin{itemize}
	\item Accommodation-Vergence: As is reported in these \cite{disconfortReview}\cite{vergenceDisconfort} articles the mismatch between accommodation and vergence in head mounted displays causes a conflict on the expected depths increasing the feeling of discomfort and dizziness.  One idea to reduce this effect is to dynamically move the position of screen frames as the focus changes from closer to distant objects and vice versa. To archive this we will use a neuronal network that will be able to discern between indoors and outdoors, also we will need the depth map obtained via the stereo camera setup. \\*
	In addition to this, a progressive change from one focus to the other may be needed, as a big change in focus can induce sickness to the user.
	
	\item Depth of field (DoF) blur: Recent investigations\cite{ifftConfortDoF} suggest that applying a DoF blur to a scene viewed using a head-mounted display can reduce visual discomfort, the challenge here is that our project has to make this DoF blur in real time in a real world environment, in contrast to the developed in that investigation that were computer generated scenes. To archive this we are going to use disparity map and the information obtained by a neural network that discerns between indoor and outdoor environments.
\end{itemize} 

Thirdly, as the headset will be used in workplaces, the idea of adding data to the environment, could improve the work-flow and the work efficiency. For this reason we think, that adding a third camera to the headset can add valuable information that can be mixed with the environment. A similar idea can be found here \cite{vismerge} where a third camera is used to add information to the real world. For example with adding a infrared camera, we can warn to the user with a mark or coloring the objects too hot to be touched.

Along all the development, user testing sessions will be done to check the improvements made between the different versions of the developed software and the different headsets developed. The main goal of this sessions will be to evaluate if the developed software works and if it reduces the sickness feelings of the users.

\section{METHODOLOGY}
Scrum and its variants are one of the most spread work methodologies nowadays, 
The methodology scrum is the selected to be used in this project, however, as this project will only be done by one person, some changes have to be made. 

First as there is only one developer de daily meeting will be substituted with a weekly meeting with the stakeholders, in this case the tutor and the boss of the laboratory department in the CVC, in these meetings we will evaluate the development done, the issues faced that week and the problems solved. In addition to that we will discuss and review future milestones and the progress towards them.

Secondly, a backlog will be prepared at the beginning of the project and will contain the main goals divided in tasks, these tasks will be organized in groups of sprints easily done in a week. As each sprint has a backlog of task to be done, a tool can be used to organize this backlog, in this case Trello\cite{trello} will be used.

Thirdly, each iteration over the documentation and the development will be kept by a version tracker, in this case, github\cite{github} and its desktop client\cite{githubDesktop}. 

Keeping the main scrum methodology, I think it's interesting to grab some ideas from the Lean software development\cite{leanMethod}. The idea of removing the so called "Muda", non important extra features and processes, could be beneficial for this project because it could reach an overwhelming dimension for the limited time that we have.

Another interesting idea is to




\section{PLANNING}
The planning timeline is on the Gantt diagram \ref{}.
\subsection{Research, documentation and planning}
This stage of the project can be split in two parts:
\begin{itemize}
	\item The first will start at the beginning and will recap information about the state of the art, do the planning and search about the different technologies involved in this project. 
	\item The second part will be happening along all the project and will consist in document every change and issue that can happen. When a part of the project changes its orientation or a milestone is archived the documentation will be updated with the changes. 
\end{itemize}
\subsection{Improving the viewer}
\subsection{Disparity map}
\subsection{Accommodation-Vergence}
\subsection{Depth of field blur}
before starting we will make a mockup of a real world environment with out of focus zones 
\subsection{Third camera}
\subsection{User Testing}

\begin{thebibliography}{11}
\bibitem{ifftConfortDoF}
CARNEGIE, Kieran; RHEE, Taehyun. Reducing visual discomfort with HMDs using dynamic depth of field. IEEE computer graphics and applications, 2015, vol. 35, no 5, p. 34-41.

\bibitem{disconfortReview}
TERZIC, Kasim; HANSARD, Miles. Causes of discomfort in stereoscopic content: a review. arXiv preprint arXiv:1703.04574, 2017.

\bibitem{vergenceDisconfort}
Hoffman, D.M., Girshick, A.R., Akeley, K. and Banks, M.S., 2008. Vergence–accommodation conflicts hinder visual performance and cause visual fatigue. Journal of vision, 8(3), pp.33-33.

\bibitem{vismerge}
ORLOSKY, Jason, et al. VisMerge: Light Adaptive Vision Augmentation via Spectral and Temporal Fusion of Non-visible Light. En Mixed and Augmented Reality (ISMAR), 2017 IEEE International Symposium on. IEEE, 2017. p. 22-31.

\bibitem{trello}
\url{https://trello.com/}

\bibitem{github}
\url{https://github.com/}

\bibitem{githubDesktop}
\url{https://desktop.github.com/}

\bibitem{leanMethod}
\url{https://msdn.microsoft.com/en-us/library/hh533841(v=vs.120).aspx}

\end{thebibliography}

\appendix

\section*{Appendix}


\end{document}

