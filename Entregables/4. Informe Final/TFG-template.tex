% !TeX spellcheck = en_US
\documentclass[10pt,a4paper,twocolumn,twoside]{article}
\usepackage[utf8]{inputenc}
\usepackage[english]{babel}
\usepackage{multicol}
\usepackage{graphicx}
\usepackage{fancyhdr}
\usepackage{times}
\usepackage{titlesec}
\usepackage{multirow}
\usepackage{lettrine}
\usepackage{pdflscape}
\usepackage{subcaption}
\usepackage{booktabs}
\usepackage[edges]{forest}
\usepackage{url}
\usepackage[top=2cm, bottom=1.5cm, left=2cm, right=2cm]{geometry}
\usepackage[figurename=Fig.,tablename=Table]{caption}
\captionsetup[table]{textfont=sc}

\author{\LARGE\sffamily Segovia Barreales, Richard}
\title{\Huge{\sffamily Reducing dizziness when using a video-see-through head-mounted display}}
\date{}

\newcommand\blfootnote[1]{%
  \begingroup
  \renewcommand\thefootnote{}\footnote{#1}%
  \addtocounter{footnote}{-1}%
  \endgroup
}

%
%\large\bfseries\sffamily
\titleformat{\section}
{\large\sffamily\scshape\bfseries}
{\textbf{\thesection}}{1em}{}

\begin{document}

	\fancyhead[LO]{\scriptsize AUTHOR: SEGOVIA BARREALES, RICHARD}
	\fancyhead[RO]{\thepage}
	\fancyhead[LE]{\thepage}
	\fancyhead[RE]{\scriptsize EE/UAB TFG INFORMÀTICA: REDUCING DIZZINESS WHEN USING A VIDEO-SEE-THROUGH HEAD-MOUNTED DISPLAY}
	
	\fancyfoot[CO,CE]{}
	
	\fancypagestyle{primerapagina}
	{
	   \fancyhf{}
	   \fancyhead[L]{\scriptsize TFG EN ENGINYERIA INFORMÀTICA, ESCOLA D'ENGINYERIA (EE), UNIVERSITAT AUTÒNOMA DE BARCELONA (UAB)}
	   \fancyfoot[C]{\scriptsize June 2018, Escola d'Enginyeria (UAB)}
	}
	
	%\lhead{\thepage}
	%\chead{}
	%\rhead{\tiny EE/UAB TFG INFORMÀTICA: TÍTOL (ABREUJAT SI ÉS MOLT LLARG)}
	%\lhead{ EE/UAB \thepage}
	%\lfoot{}
	%\cfoot{\tiny{February 2015, Escola d'Enginyeria (UAB)}}
	%\rfoot{}
	\renewcommand{\headrulewidth}{0pt}
	\renewcommand{\footrulewidth}{0pt}
	\pagestyle{fancy}
	
	%\thispagestyle{myheadings}
	\twocolumn[\begin{@twocolumnfalse}
	
	%\vspace*{-1cm}{\scriptsize TFG EN ENGINYERIA INFORMÀTICA, ESCOLA D'ENGINYERIA (EE), UNIVERSITAT AUTÒNOMA DE BARCELONA (UAB)}
	
	\maketitle
	
	\thispagestyle{primerapagina}
	%\twocolumn[\begin{@twocolumnfalse}
	%\maketitle
	%\begin{abstract}
	\begin{center}
	\parbox{0.915\textwidth}
	{\sffamily
	\textbf{Resum--}
	Resum del projecte, màxim 10 línies. ........ ........... .......... ..  ... ..... .... ........ ........... .......... ..  ... ..... .... ........ ........... .......... ..  ... ..... .... ........ ........... .......... ..  ... ..... .... ........ ........... .......... ..  ... ..... .... ........ ........... .......... ..  ... ..... .... ........ ........... .......... ..  ... ..... .... ........ ........... .......... ..  ... ..... .... ........ ........... .......... ..  ... ..... .... ........ ........... .......... ..  ... ..... .... ........ ........... .......... ..  ... ..... .... .................. ..  ... ..... .... ........ ........... .......... ..  ... ..... .... ........ ........... .......... ..  ... ..... .... ........ ........... .......... ..  ... ..... .... ........ ........... .......... ..  ... ..... .... ........ ........... .......... ..  ... ..... .... ........ ........ .......... ..  ... . ........... .......... ..  ... ..... .... ........ ........... .......... ..  ... ..... .... ........ ........... .......... ..  ... ..... .... ........ ........... .......... ..  ... ........... ..  ... ..... .... ........ ........... .......... ..  ... ..... .... ........ ........... .......... ..  ... ..... .... ........ ........... .......... ..  ... ..... .... ........ ........... .......... ..  ... ..... .... ........ ........... .......... ..  ... ..... .... ........ ........... .......... ..  ... ..... .... ........ ........... .......... ..  ... ..... .... ........ ........... .......... ..  ... ..... .... 
	\\
	\\
	\textbf{Paraules clau-- } Paraules clau del treball, màxim 2 línies . .... ........ ........... .......... ..  ... ..... .... ........ ........... .......... ..  ... ..... .... ........ ........... ................\\
	\\
	%\end{abstract}
	%\bigskip
	%\begin{abstract}
	\bigskip
	\\
	\textbf{Abstract--} Versió en anglès del resum . ........ ........... .......... ..  ... ..... .... ........ ........... .......... ..  ... ..... .... ........ ........... .......... ..  ... ..... .... ........ ........... .......... ..  ... ..... .... ........ ........... .......... ..  ... ..... .... ........ ........... .......... ..  ... ..... .... ........ ........... .......... ..  ... ..... .... ........ ........... .......... ..  ... ..... .... ........ ........... .......... ..  ... ..... .... ........ ........... .......... ..  ... ..... .... ........ ........... .......... ..  ... ..... .... .................. ..  ... ..... .... ........ ........... .......... ..  ... ..... .... ........ ........... .......... ..  ... ..... .... ........ ........... .......... ..  ... ..... .... ........ ........... .......... ..  ... ..... .... ........ ........... .......... ..  ... ..... .... ........ ........ .......... ..  ... . ........... .......... ..  ... ..... .... ........ ........... .......... ..  ... ..... .... ........ ........... .......... ..  ... ..... .... ........ ........... .......... ..  ... ........... ..  ... ..... .... ........ ........... .......... ..  ... ..... .... ........ ........... .......... ..  ... ..... .... ........ ........... .......... ..  ... ..... .... ........ ........... .......... ..  ... ..... .... ........ ........... .......... ..  ... ..... .... ........ ........... .......... ..  ... ..... .... ........ ........... .......... ..  ... ..... .... ........ ........... .......... ..  ... ..... .... 
	\\
	\\
	\textbf{Keywords-- } Versió en anglès de les paraules clau. .... ........ ........... .......... ..  ... ..... .... ........ ........... .......... ..  ... ..... .... ........ ........... .................. ..\\
	}
	
	\bigskip
	
	{\vrule depth 0pt height 0.5pt width 4cm\hspace{7.5pt}%
	\raisebox{-3.5pt}{\fontfamily{pzd}\fontencoding{U}\fontseries{m}\fontshape{n}\fontsize{11}{12}\selectfont\char70}%
	\hspace{7.5pt}\vrule depth 0pt height 0.5pt width 4cm\relax}
	
	\end{center}
	
	\bigskip
	%\end{abstract}
	\end{@twocolumnfalse}]
	
	\blfootnote{$\bullet$ E-mail: richard.segovia@e-campus.uab.cat}
	\blfootnote{$\bullet$ Menció en Computació}
	\blfootnote{$\bullet$ Project supervised by: Coen Antens (CVC) and Felipe Lumbreras (Computació)}
	\blfootnote{$\bullet$ Course 2017/18}
	
	\section{Introduction}
	%- introduccion al vr
	%- explicar motivacion, practicas de verano, procarelight, usuarios mareados
	%- explicar projecto en si, idea
	%- explicar Accomodation vergence con un dibujo y de forma resumida
	
	The development of the technologies related with head mounted displays (HMD) has grown in the recent years mainly centered in the video-games field, some examples are the Oculus \cite{web:oculus} or the HTC Vive \cite{web:vive}. These devices are called virtual reality headsets because they are only capable of showing computer generated scenes.  
	
	Despite that the industry is mainly focused developing virtual reality systems, there are two kinds of HMD that allow the visualization of the environment surrounding the user.  

	\begin{itemize}
	\item The optical-see-through use image projectors that display the image over a see-thought mirror, hence allowing the user to see computer generated images over the environment. An example of this kind of devices is the Microsoft HoloLens \cite{web:hololens}.
	\item The video-see-through devices use one or two cameras placed in the front of the headset and show the stream of images in two screens places in front of the eyes. This project uses these kind of devices, one of the prototypes can be seen in Fig.\ref{fig:proto}. 
	\end{itemize}

	\begin{figure}
		\centering
		\includegraphics[width=1\linewidth]{img/imagenproto3.jpg}
		\caption{Current version of the HMD video-see-through prototype.}
		\label{fig:proto}
	\end{figure}
	
	Both types of devices can show stereo images, this allow the user to sense depth in the displayed images. 
	
	This project originated from the need to establish and resolved the reasons why the users have a poor experience, dizziness and eye strain, when using video-see-through devices. As is explained in \cite{disconfortReview} and was experimentally tested in \cite{vergenceDisconfort}, the Accommodation-Vergence conflict is one of the issues that causes a poor user experience.
	
	The vergence is the process where the eyes set their angle of visualization trying to fuse the image of an object keeping it into sharp focus, whereas the accommodation is the process where the objects difficult to fuse are blurred. Both process are tightly coupled giving each other feedback in order to keep the images as sharp and fused as possible. 
	
	The issue arises as a result of using near eye screens to show stereo 3D. In these displays the image is shown always at the same distance of the eye, however, the distance between objects, disparity,  changes when the environment is changed. This conflict can be seen in Fig.\ref{fig:vergence}
	
	\begin{figure}
		\centering
		\includegraphics[width=1\linewidth]{img/vergencia.png}
		\caption{ FALTA EXPLICAR Image from \cite{vergenceDisconfort}.}
		\label{fig:vergence}
	\end{figure}
	
	
	Consequently, this project tries to solve the Accommodation-Vergence conflict using the environment changes information to set the distance between images of the viewer. 
	
		
	\section{State of the art}
	%explicar de donde viene el vr, hacia donde va, tecnicas tipicas usadas para resolucion del problema de la vergencia (seguimiento de los ojos), hablar de tecnicas de depth map
	%- hardware que se utiliza ahora, problema y incovenientes
	%- depth map, Libelas, explicar que existe la middlebury datebase y porque hemos elegido libelas???????????????????
	%- calibration (?) explicar en que consiste
	The Accomodation-Vergence conflict is a topic of great interest in the research field, therefore, a wide variety of  solutions have been proposed to try to solve this problem. 
	
	One of the many solutions found to ease this problem is applying blur to zones where the image should not fuse, thus simulating the effect that the image would have had if the conflict would not have happened, see \cite{neareyeblur}. Related with this, a research \cite{sceneComposition} found that placing objects that connect different depth planes helps the users to better transition between objects in different planes and aids to maintain the coupling between accommodation and vergence. Other line of research is to use eye-tracking techniques \cite{eyeTracking}, these technologies allow the system to know where the eye is pointing over the image and then blur the out of focus areas. 
	
	After all of this, we can see that this project takes a different approach from the main lines of research.  This project uses depth data collected from a stereo camera placed in front of the headset to change the distance between images inside the headset. However, to get the depth information a reliable and fast stereo matcher was needed. That is why ELAS \footnote{Efficient large-scale stereo matcher} \cite{LIBELAS} and its implementation LIBELAS were used.  
		
	In the stereo matcher field, two main branches can be found, Local stereo matchers and global stereo matchers. Global stereo matchers are reliable but use much compute power. On the contrary, local stereo matchers are fast but less reliable than the global matchers. ELAS uses first a global stereo matcher on highly reliable points and after removing the more redundant, a Delaunay triangle mesh is created from that points. After that, the regions are then processed by a local stereo matcher. More information about ELAS implementation can be found on \cite{LIBELAS}. 

	\section{Objectives}
	After the previous analysis and explanation of the problem, the main objectives will be the following: 

	\begin{itemize}
		\item Evaluate the user experience when using the HMD and determine whether the accommodation-vergence effect causes dizziness and general discomfort on the users. 
		
		\item If the Accomodation-Vergence is one of the causes of the discomfort on the users, the problem will be solved using the deph information that can be obtained using stereo vision. 
		
		\item Evaluate the user experience after the development and conclude if our approach for solving this problem has reduce the discomfort on the users. 
		
		\item Add the required modules keeping in mind the usefulness of these for future developments.
		
		\item Make all the required changes in the visualization program without reducing the performance. 
	\end{itemize}

	\section{Methodology}
	%dudas sobre este apartado
	%hablar de tecnicas de desarrollo agil y sobretodo de las librerias utilizadas y de las limitaciones y ventajas que estas nos han aportado.
	
	Scrum and its variants are one of the most spread work methodologies nowadays. Therefore we believe convenient to use it as one of the foundation of this project, however, as this project was only done by one person, some changes were made. 
	
	As there was only one developer, the daily meeting was replaced with a weekly meeting with the project supervisors, in this case the tutor and the boss of the laboratory department in the CVC. In these meetings we evaluated the development done, the issues faced that week and the problems solved. Related with the scrum methodology, Github \cite{web:github},\cite{web:githubDesktop} was used as version control system and trello\cite{web:trello} as the task manager system.
	
	Involving the tools used to develop this project, C++ and QT \cite{web:qt} were used mainly because the previous development of the visualization system was done using that environment. In addition to that, OpenCV was used as the library for image processing and calibration, see REF!!.
	
	
	
	
	\section{Tools and Development}
	
	este apartado hablare del codigo desarrollado, los problemas encontrados y las soluciones realizadas
	
	\subsection{Calibration}
	en este apartado hablare sobre que tecnicas se han usado para realizar la calibracion y como se ha implementado y estructurado, explicar problemas encontrados para la calibracion (descalibracion constante parametros utilizados)
	
	\subsection{Libelas}
	 como se utiliza y que necesita para obtener la profundidad (imagenes epipolares, texturas heterogeneas etc.), problemas de sensibilidad a la calibracion, 
	
	\subsection{Dataset}
	%grabacion, y captura del dataset
	en este apartado se hablara del modulo de grabacion, y el pipeline paralelo (sin usar el viewer) que se montó para hacer funcionar el sistema en tiempo real y poder realizar los analisis.
	
	\subsection{Integration with the viewer}
	%imagen del diseño del pipeline final, explicar clasificador 
	%presets, movimiento de las pantallas , guardado de parametros, integracion de las partes, threads
	en este apartado se explicaran mejoras secundarias en el visor, creacion de los presets, smooth presets transition, movimiento de la roi por encima de las imagenes etc, y de como se han integrado todos los modulos dentro de visor intentando preservar en todo momento el rendimiento (threads, explicacion del pipeline final)

	
	\section{Results}
	
	%\subsection{Calibration Comparison}
	%comparaciones entre imagenes no calibradas y imagenes calibradas
	%en este apartado compararemos las imagenes originales con calibracion obtenida via opencv y la obtenida via %matlab (quizas es un poco innceseraio este apartado ?)
	
	\subsection{Libelas}
	%rendimiento fps y resultados en cuanto a profundidad. imagenes!!
	en este apartado de evaluaran los resultados con distintas resoluciones (subsampling) tanto en rendimiento puro (fps) como en resultados visuales de la profundidad (calidad de la dispariedad obtenida). tambien se vera una comparativa entre superficies/objetos con textura variada y sin textura
	
	\subsection{First user testing}
	aqui expondremos los resultados de la primera prueba de user testing que se hizo, explicaremos nuestras conclusiones previas sobre los resultados de la prueba y nuestras propuestas para tratar de mejorar los resultados
	
	\subsection{Second user testing}
	en este apartado mostraremos los distintos resultados obtenidos en la segunda sesion de user testing , explicaremos los resultados y concluiremos si nuestra hipotesis es correcta y si la solucion desarrollada es suficiente para resolver este problema, en caso de que no, nos plantearemos cuales son o han sido los problemas que impiden que el usuario sienta una mejora al utilizar la vergencia dinamica.
	
	\section{Conclusions}
	%Parte importante!
	Finalmente expondremos todo el trabajo realizado y apartir de los resultados de las sesiones de user testing explicaremos si se han logrado los objetivos y cuanto margen de mejora hay en caso de haberlo. 
	
	\section{Future work}
	%cosas que se han quedado en el cajon
	(quizas esto es mas para la presentacion, nose si en el informe tambien se deberia de poner)
	en este apartado explicaremos ideas que se plantearon y que no llegaron a realizarse y ideas con las que podria continuarse este proyecto, (DoF blur, third camera, Augmented reality, sistemas mas pequeños (voyo)  )
	
	\section*{Acknowledgment}
	This work is supported in part by a CVC transfer project with ProCare Light company, and partially funded by the Spanish Ministry of Economy and Competitiveness and FEDER under grants TIN2014-56919-C3-2-R and TIN2017-89723-P.
	
	\bibliography{biblio}
	\bibliographystyle{plain}
	
	\appendix
	
	\section{Objective and tasks list}
	lista de tareas y objetivos (similar a lo que tenia en las otras entregas)
	
	\section{Additional images}
	%añadir imagenes de oftalmologia
	En este apartado se incluiran imagenes extra de ejemplo(mas escenarios) i/o imagenes que no quepan en el documento en si
	
	\section{Gantt planning}
	el diagrama de gantt final (quizas esto va en el dossier en vez de aqui?)


\end{document}

